% Project Proposal LaTeX doc:

\documentclass[12pt,conference]{IEEEtran}
\usepackage{mathtools}

\usepackage{algorithm}% http://ctan.org/pkg/algorithms
\usepackage{algpseudocode}% http://ctan.org/pkg/algorithmicx

\usepackage{graphicx}
\usepackage{amsmath}

\usepackage{stfloats}

\graphicspath{ {images/} }

\hyphenation{op-tical net-works semi-conduc-tor}

\begin{document}
\raggedbottom

\title{Bounding the Genus of a Graph: A Project Proposal}

\author{\IEEEauthorblockN{Taylor Cox}
\IEEEauthorblockA{Department of Computer Science\\
University of Manitoba\\
Winnipeg, Manitoba\\
Email: coxt3@myumanitoba.ca}}

\maketitle

\begin{abstract}

In 1989, Thomassen proved that the graph genus problem is NP-Complete. However, in 1999 Mohar gave a linear-time algorithm for determining whether a graph can be embedded on a specific surface. The goal of this project is to investigate these papers and others to evaluate the current state of the graph genus problem. Approaches will then be developed to bound the genus of a graph G to a sufficiently small interval such that repeated execution of Mohar 1999 will be feasible. Therefore, the expected result of this project is a set of one or more heuristics that will reduce the search space for determining the genus of a graph.

\end{abstract}

\begin{IEEEkeywords}
Genus,Embedding,Surfaces,Bounding
\end{IEEEkeywords}

\section{Introduction}

Problems in graph embedding lie at the intersection between Graph Theory and Topology. The embedding problem of specific interest to this project is the Graph Genus Problem. The Graph Genus Problem (GGP) is as follows: Given a graph G and a natural number k, determine whether G has a genus of k or less. A graph has genus q if G can be embedded on some surface $\Sigma$ with genus q. The genus of $\Sigma$ is the number of handles that exist on $\Sigma$ in cases where $\Sigma$ is an orientable surfaces, or the number of crosscaps in cases where $\Sigma$ is unorientable. In cases of unorientability, the genus is also referred to as the \textit{demigenus}. The attribute of particular interest to the graph genus problem is the minimum genus of G. In fact, the genus of G normally denotes the minimum genus of G. 

In 1979, Filotti et al. gave a polynomial time algorithm for solving the GGP for a fixed value of k \cite{filotti}. Ten years later, Thomassen proved that the general GGP is NP-Complete \cite{thomassen}. In 2011, Myrvold et al. identified a fatal flaw in Filotti 1979, proving the algorithm's incorrectness \cite{myrvold-kocay}. However, Mohar was able to give a linear time algorithm for the fixed-k GGP in 1999 \cite{mohar}. Despite its difficult nature, Mohar 1999 has not been disproven. 

The goal of this project is to leverage Mohar 1999 for the purpose of determining the genus of a graph with a non-fixed k. This does not mean that this project intends to prove the GGP is in P. The intension of this project is to develop heuristic approaches to bound the genus of a graph to a sufficiently small range such that repeated execution of Mohar 1999 will be feasible. The specific goals of this project are as follows:

\begin{enumerate}
\item Evaluate the current state-of-the-art with respect to the GGP
\item Investigate the theoretical bounds for the genus of a graph
\item Determine methods for ruling-out or ruling-in the genus of a graph using Graph Minor Theory
\end{enumerate}

Part (1) will be accomplished by conducting a thorough literature survey, with initial points of investigation identified further in this work. Specific works in the GGP literature will provide insight into part (2). Part (2) will use tools including the Euler-Poincair\'e formula and theoretical findings from authors including Xuong 1979 \cite{xuong} to limit the search space for determining the genus of a graph. Finally, part (3) will use existing findings in Graph Minor Theory to further limit the search space for a graph's genus. This project will capitalize on findings similar to Kuratowski's theorem to further ascertain which surfaces a graph can or cannot be embedded on.

The remainder of this proposal is structured as follows: Section II gives a detailed definition of graph embedding and the GGP. Section III constitutes a survey of literature relevant to graph genus bounding, graph minor theory, and the GGP itself. Section IV will then further discuss the relevance of graph minor theory and graph genus bounding to this project, and section V will describe the contributions this project will make based on graph minor theory and graph genus bounding. Finally, section VI will detail the project's proposed programming components.

\section{Problem Statement}

In order to accurately describe the GGP, a series of nonelementary definitions must first be laid out. This proposal omits elementary graph theory definitions.
Recall that the GGP is as follows: Given a graph G and a natural number k, determine whether g(G) $\leq$ k, where g(G) denotes the genus of G. g(G) is the genus of the minimally-genal surface G can be embedded on. A graph G is considered embeddable on some surface $\Sigma$ if there exists a drawing of G on $\Sigma$ such that the edges of G intersect only at their endpoints, namely the vertices of G. Embeddings of particular importance to this work and to the general GGP literature are \textit{2-cell} embeddings. An embedding of G on $\Sigma$, denoted $G^{\psi}$ is a 2-cell embedding if every face of $G^{\psi}$ is a 2-cell. That is, every face of $G^{\psi}$ is homeomporphic (topologically equivalent) to an open disc. For the remainder of this project, only 2-cell embeddings will be of interest. 

% next: surface genus
% touch on orientability
% define the problem itself as needed
% determine citations

% Informally, the genus of G g(G) is the minimum number of handles that must exist on some surface $\Sigma$ in order for G to be embeddable on $\Sigma$. Formally, g(G) is the maximum number of non-intersecting Jordan Curves 

\section{Relevant Literature}

\section{Areas of Investigation}

\section{Proposed Contributions}

\section{Programming Components}

\section{Conclusion}

\begin{thebibliography}{1}
\bibitem{filotti}
\bibitem{thomassen}
\bibitem{myrvold-kocay}
\bibitem{mohar}
\bibitem{xuong}
\end{thebibliography}

\end{document}

% Problems related to Graph Embedding lie at the intersection of Graph Theory and Topology. Graph Embedding problems ask questions concerning particular graphs and the particular surfaces they may be embedded on. An \textit{embedding} of a graph G on a surface $\Sigma$ is a drawing of G on $\Sigma$ where the edges of G may only intersect at the vertices of G \cite{define-embedding}. A specific example of a graph embedding problem is the problem of graph planarity. The graph planarity problem asks whether a given graph G may be embedded on the plane. While the graph planarity problem is solved by Kuratowski's theorem from \cite{kuratowski} in conjunction with the Hopcroft-Tarjan algorithm from \cite{hopcroft-tarjan}, it is not possible to generalize these solutions to other surfaces. For most surfaces, there is no complete set of minors nor is there a polynomial time to decide


% Graph Embedding Problems lie at the intersection between Graph Theory and Topology. The problem in graph embedding of particular interest to this project is the graph genus problem, defined as follows: Given a Graph G = (V,E), find the minimum genus of G. The genus of a surface $\Sigma$ is the number of handles on the surface in the case of orientable surfaces, or the number of crosscaps in the case of non-orientable surfaces. Therefore, the genus of a graph G is the minimum number of handles or crosscraps that must be present on the surface in order for an embedding of G to be possible. An embedding E of G on $\Sigma$ is a drawing of G on $\Sigma$ where the edges of G only intersect at their endpoints, namely the vertices of G. In 1989, Thomassen proved that the graph genus problem is NP-Complete. However, in 1979 Filotti et al. gave a fixed paramater-tractable algorithm for the graph genus problem. The goal of this project is to investigate these papers and others to evaluate the current state of the graph genus problem. Approaches will then be developed to bound the genus of a graph G to a sufficiently small interval such that repeated execution of the algorithm in Filotti 1979 will be feasible.


% Graph Embedding Problems lie at the intersection between Graph Theory and Topology. The problem in graph embedding of particular interest to this project is the graph genus problem, defined as follows: Given a Graph G = (V,E), find the minimum genus of G. 

% The remainder of this proposal is structured as follows: Section II gives a detailed definition of the graph genus problem. Section III constitutes a survey of literature relevant to the GGP. Next, section IV will discuss specific points of investigation. Section V will propose contributions related to those points, and section VI will detail the proposed programming components of this project.