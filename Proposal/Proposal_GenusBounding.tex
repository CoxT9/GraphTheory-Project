% Project Proposal LaTeX doc:

\documentclass[12pt,conference]{IEEEtran}

\usepackage{mathtools}
\usepackage{algorithm}
\usepackage{algpseudocode}
\usepackage{graphicx}
\usepackage{amsmath}
\usepackage{stfloats}

\graphicspath{ {images/} }

\hyphenation{op-tical net-works semi-conduc-tor}

\begin{document}
\raggedbottom

\title{Bounding the Genus of a Graph: A Project Proposal}

\author{\IEEEauthorblockN{Taylor Cox}
\IEEEauthorblockA{COMP 4060: Graph Theory\\
University of Manitoba\\
Email: coxt3@myumanitoba.ca}}

\maketitle

\begin{abstract}

In 1989, Thomassen proved that the graph genus problem is NP-Complete. However, in 1999 Mohar gave a linear-time algorithm for determining whether a graph can be embedded on a specific surface. The goal of this project is to investigate these papers and others to evaluate the current state of the graph genus problem. Approaches will then be developed to bound the genus of a graph $\gamma$(G) to a sufficiently small interval such that repeated execution of Mohar 1999 will be feasible. Therefore, the expected end result of this project is a set of one or more heuristics that will reduce the search space for determining the genus of a graph.

\end{abstract}

\begin{IEEEkeywords}
Genus,Embedding,Surfaces,Bounding
\end{IEEEkeywords}

\section{Introduction}

Problems in graph embedding lie at the intersection between Graph Theory and Topology. The graph embedding problem of specific interest to this project is the Graph Genus Problem. The Graph Genus Problem (GGP) is as follows: Given a graph G and a natural number k, determine whether G has a genus of k or less. A graph has genus $\gamma$ if G can be embedded on some surface $\Sigma$ of genus $\gamma$. The genus of $\Sigma$ is the number of handles that exist on $\Sigma$ in cases where $\Sigma$ is an orientable surface, or the number of crosscaps in cases where $\Sigma$ is unorientable. In cases of unorientability, the genus is also referred to as the \textit{demigenus}. The attribute of particular interest to the graph genus problem is the minimum genus of G. In fact, the genus of G normally denotes the minimum genus of G. 

In 1979, Filotti et al. gave a polynomial time algorithm for solving the GGP for a fixed value of k \cite{filotti}. Ten years later, Thomassen proved that the general GGP is NP-Complete \cite{thomassen}. In 2011, Myrvold et al. identified a fatal flaw in Filotti 1979, proving the algorithm's incorrectness \cite{myrvold-kocay}. However, Mohar was able to give a linear time algorithm for the fixed-k GGP in 1999 \cite{mohar}. Despite its complicated nature, Mohar 1999 has not been disproven. 

The goal of this project is to leverage Mohar 1999 for the purpose of determining the genus of a graph with a non-fixed k. This does not mean that this project intends to prove the GGP is in P. The intension of this project is to develop heuristics for bounding the genus of a graph to a sufficiently small range such that repeated execution of Mohar 1999 will be feasible. The specific goals of this project are as follows:

\begin{enumerate}
\item Evaluate the current state-of-the-art with respect to the GGP
\item Investigate the theoretical bounds for the genus of a graph
\item Determine methods for ruling-out or ruling-in the genus of a graph using Graph Minor Theory
\end{enumerate}

Part (1) will be accomplished by conducting a thorough literature survey, with initial points of investigation identified further in this work. Specific works in the GGP literature will provide insight into part (2). Part (2) will use tools such as the Euler-Poincair\'e formula $v-e+f=2-2g$ and theoretical findings from authors including Xuong 1979 \cite{xuong} to limit the search space for determining the genus of a graph. Finally, part (3) will use existing findings in Graph Minor Theory to impose additional search space limits. This project will capitalize on findings similar to Kuratowski's theorem to further ascertain which surfaces a graph can or cannot be embedded on.

The remainder of this proposal is structured as follows: Section II gives a detailed definition of graph embedding and the GGP. Section III constitutes an overview of literature relevant to graph genus bounding, graph minor theory, and the GGP itself. Section IV will then further discuss the relevance of graph minor theory and theoretical graph genus bounding to this project, and section V will detail the expected results this project will achieve based on graph minor theory and theoretical graph genus bounding. Finally, section VI will detail the project's proposed programming components.

\section{Problem Statement}

In order to accurately describe the GGP, a series of nonelementary definitions must first be laid out. This proposal omits elementary graph theory definitions.
Recall that the GGP is as follows: Given a graph G and a natural number k, determine whether $\gamma$(G) $\leq$ k, where $\gamma$(G) denotes the genus of G. $\gamma$(G) is equal to the genus of the minimally-genal surface G can be embedded on. A graph G is considered embeddable on some surface $\Sigma$ if there exists a drawing of G on $\Sigma$ such that the edges of G intersect only at their endpoints, namely the vertices of G. 

Embeddings of particular importance to this work and to the general GGP literature are \textit{2-cell} embeddings. An embedding of G on $\Sigma$, denoted $G^{\psi}$ is a 2-cell embedding iff every face of $G^{\psi}$ is a 2-cell. That is, every face of $G^{\psi}$ is homeomporphic (topologically equivalent) to an open disc \cite{gao}. For example, an embedding of $K_{4}$ on the torus is not a 2-cell embedding, because one of the faces is homeomorphic to a perforated disk. For the remainder of this project, only 2-cell embeddings will be of interest. 

To determine the genus of a graph, it is first necessary to define the term genus itself. As mentioned, the genus of a graph G is determined by the genus of the surface G is to be embedded upon. Determining the genus of some surface $\Sigma$ first depends on whether $\Sigma$ is orientable. A surface $\Sigma$ is considered orientable if it has two sides. Informally, $\Sigma$ is orientable if a pointer can be continuously moved about the surface without causing the pointer to become inverted. For example, the sphere is an orientable surface because no pointer may continuously travel about the sphere and manage to enter the sphere itself. In contrast, the M\"obius band is an \textit{un}orientable surface because a pointer may continuously travel about the M\"obius band and eventually arrive at the same location in the band, on the opposite ``side''.
  
In cases where $\Sigma$ is an orientable surface, the genus of $\Sigma$ is the number of handles attached to $\Sigma$. For example, a handle attached to a surface of genus 0 (namely the Sphere) causes the surface to exhibit a donut shape (the Torus). Moreover, the genus of $\Sigma$ for some unorientable $\Sigma$ is equal to the number of crosscaps attached to $\Sigma$. Similar to the case where $\Sigma$ is orientable, we begin with the sphere. A sphere with no crosscaps has an unorientable (and orientable) genus of 0. A crosscap is homeomorphic to a M\"obius band. Therefore, the genus of an unorientable surface is equivalent to the number of M\"obius bands sewn onto a sphere. The genus of a surface can also be defined formally using Jordan Curves. That is, the genus of $\Sigma$ is equal to the number of non-intersecting Jordan Curves that may be drawn on $\Sigma$ without causing a division into multiple pieces \cite{gao}. The Jordan Curve definition of genus results in the same values as the handle or crosscap variants. 

The GGP may now be described specifically. The GGP aims to answer the following question: Given a graph G, determine the minimum number of handles that must be added to some surface $\Sigma$ in order for a 2-cell embedding of G to be possible on $\Sigma$. In cases where $\Sigma$ is unorientable, the GGP remains the same, except ``crosscaps'' replaces ``handles'' where used. The GGP is known to be NP-Complete \cite{thomassen} but is also known to be fixed-parameter tractable \cite{mohar}. Therefore, while it is unrealistic to develop an algorithm to solve the GGP, efforts may be made to reduce the number of possibilities for a graph's genus. If the bounds on a graph's genus are sufficiently narrow, feeding the reduced possibilities to a fixed-paramater algorithm may become a viable option for determining $\gamma$(G).

\section{Relevant Literature}

The development of a comprehensive approach to bounding the genus of a graph as tightly as possible has recieved little attention in the graph theory literature. Xuong \cite{xuong} and Archdeacon et al. \cite{archdeacon} have been able to introduce upper bounds on the genus of a graph using the Betti Number of a graph B(G). In addition, authors including Garagin, Myrvold and Chambers \cite{gagarin-myrvold} and Thomas \cite{thomas} have completed works to further characterize the genus of a graph using Graph Minor Theory. Graph Minor Theory itself is the result of Robertson and Seymour's extension of Kuratowski's theorem to graphs on other surfaces. A thorough exposition of Graph Minor Theory can be found in Lov\'asz 2005 \cite{lovasz}. The aspect of Graph Minor Theory of particular interest is the eighth paper of Robertson and Seymour's series on graph minors, where the authors extend the Kuratowski theorem to any general surface \cite{graph-minors-8} Despite the extensive work understaken in areas relevant to the GGP, little work has been accomplished that encompasses all known methods of graph genus bounding in pursuit of a robust answer to the GGP. 

The GGP was first proven to be NP-Complete by Thomassen in \cite{thomassen} by showing that the GGP is polynomial time reducible to the maximal independent set problem. Despite Thomassen's proof, a number of fixed-k GGP algorithms have been developed, including those by Mohar \cite{mohar} and by Filotti et al. \cite{filotti}. Myrvold and Kocay proved in 2011 that the algorithm presented by Filotti et al. was incorrect \cite{myrvold-kocay}. However, Mohar's algorithm remains as an option to determine whether a graph G may be embedded on a specific surface $\Sigma$.

Graph embedding algorithms themselves have also been covered extensively in the literature. Graph embedding algorithms are a crucial aspect of approaching the GGP as an embedding of a graph G on a surface $\Sigma$ must be shown in order to prove such an embedding exists. Graph embedding algorithms also determine whether a graph is embeddable on specific surfaces. However, these algorithms do not apply to all surfaces in general. Kocay et al. extended Read's algorithm for drawing planar graphs to toroidal graphs in 2001 \cite{kocay}. In 2003, Gagarin gave a linear time algorithm to detect if a graph can be drawn in the projective plane \cite{gagarin}. 

The primary literary baselines of this project are expected to be Mohar's linear-time fixed-k GGP algorithm, Xuong's upper bound of $\gamma$(G) based on the Betti Number B(G), and Robertson and Seymour's development of Graph Minor theory to generalize Kuratowski's theorem to any surface. The combined results from these separate works is expected to result in a substantial set of approaches that will reduce the possibilities for the genus of a given graph.

\section{Areas of Investigation}

The key areas of investigation with respect to this project are motivated by the pursuit of the three points laid out in section I. The investigation of the current literatute, the determination of mathematical bounds for $\gamma$(G), and the determination of bounds for $\gamma$(G) based on Graph Minor Theory together composes the work to be undertaken during this project. The execution of each project aspect is intended to (1) further the understanding of the GGP and (2) pursue new methods for aiding in the determination of $\gamma$(G). The goal of this project is not to solve the GGP in polynomial timeside. Rather, this project aims to develop methods for limiting the search space for determining $\gamma$(G).

\subsection{Continued Literature Review}

An expansion of the literature survey outlined in this proposal is the first significant aspect of work to be accomplished during this project. Section III of this work describes existing literature relevant to the ``core'' GGP, along with an outline of developments in approaches for bounding $\gamma$(G). The next phase of the literature survey will engage in a more thorough examination of graph genus bounding approaches. The current literature regarding graph genus maximization will be surveyed and deployed for use in the implementation aspect of this project. The same steps apply to Graph Minor Theory. This project will conduct a thorough examination of Robertson and Seymour's work as well as other contributions to precisely determine the relationship between specific graph minors and specific surfaces. While the collection of forbidden minors of all surfaces is not complete \cite{thomas}, a key goal of the continued literature review is to capture a list of forbidden graph minors that is as complete as possible.

% A collection of forbidden and permissable graph minors will be recorded for use in the torus, double-torus, M\"obius band, Klein bottle, and other surfaces of interest. Collecting such forbidden minors is intended to provide a comprehensive list of minors to determine the embeddibility of a given graph G on any surface. This approach is analogous to 

\subsection{Graph Genus Bounding}

Methods for graph genus bounding will be determined and implemented in tandem with the completion of the literature survey described. These methods concern approaches drawn from areas other than Graph Minor Theory. In particular, this project will examine the use of the Euler-Poincair\'e formula and the Betti Number B(G) to maximize $\gamma$(G) as described in Xuong 1979 and Archdeacon 2014. In Xuong 1979, the maximal genus of a graph $\gamma$m(G) is stated as the following equation:
\begin{align*}
\gamma m(G) = \frac{1}{2}(\beta(G) - \mathcal{E}(G))
\end{align*}

Where $\beta$(G) denotes the Betti Number of G and $\mathcal{E}$(G) denotes the \textit{Betti Deficiency} of G. The Betti Number of a graph G is defined as $1 - v + e$, where $v$ and $e$ denotes the number of vertices and edges of G. The Betti Deficiency of a graph G is is the minimum deficiency (distance from a 1-face embedding \cite{archdeacon}) of all possible embeddings of G. Nebesk\'y (1981) defines $\mathcal{E}$(G) as:
\begin{align*}
&\ \mathcal{E}(G) =  max(u (G,A) : A \subseteq E(G)) \ \ \ \ \ \ \ \cite{archdeacon} \cite{nebesky}\\
&\ u(G,A) = ec(G-A) + 2oc(G-A) - |A| - 1 &\
\end{align*} 

Where ec(G-A) denotes the number of components of G with an even Betti Number, and oc(G-A) denotes the number of components with odd Betti Number. A denotes a subset of edges G in e. Determining A as well as deriving the proof for the $\gamma$m(G) is conducted in Archdeacon 2014 \cite{archdeacon}. This project will employ $\gamma$m(G) to bound $\gamma$(G) in anticipation of further bounding introduced with Graph Minor Theory.

\subsection{Graph Minor Theory}

The second approach for bounding $\gamma$(G) is derived from Graph Minor Theory. Graph Minor Theory offers a means to rule-in or rule-out potential surfaces a graph G may be embedded on. Graph Minor Theory may be summarized by the following statement: Given an infinite set of finite graphs, it is guaranteed that one graph in the set is a minor of another. This means that all graphs have some minor, permitting the use of Kuratowski's theorem and extended versions for graph embedding problems on any surface. Kuratowski's theorem states that a graph G is planar iff G has no minor $TK_{5}$ or $TK_{3,3}$. In order words, $TK_{5}$ and $TK_{3,3}$ are \textit{forbidden minors} for the plane. Graph Minor Theory extends Kuratowski's theorem to other surfaces. The result of this extension is that for any surface $\Sigma$, there exists a set of forbidden minors which no graph G embedabble in $\Sigma$ may contain any constituents of. Let M($\Sigma$) denote the complete set of forbidden minors for some surface $\Sigma$. M($\Sigma$) has been discovered for surfaces such as the projective plane as seen in Mohar and Thomassen (2001) \cite{mohar-thomassen} and for the plane itself, by Kuratowski's theorem. However, M($\Sigma$) is not known for any arbitrary surface. For example, there are hundreds of thousands of forbidden minors for the Torus, and the list is not known to be complete \cite{gagarin-myrvold}. 

This project will use Graph Minor Theory to limit the number of possible surfaces a graph G can be embedded on, in pursuit of determining $\gamma$(G). There exist multiple polynomial-time algorithms for determining if a graph G contains a specific minor H, as given by Kawarabayashi $(O(n^2))$ \cite{kawarabayashi} and by Robertson and Seymour $(O(n^3))$ \cite{graph-minors-13}. Therefore, determining whether a graph G contains a forbidden minor that will prevent the embedding of G on some surface $\Sigma$ is a problem solvable in polynomial time. A key issue to be addressed in this project is the matter of scaling a forbidden minor-searching algorithm to all types of minors for any arbitrary surface. This project will use results from authors including Gagarin et al. \cite{gagarin-kocay} to reduce the total number of possible minors to a smaller set of key characteristics.

\section{Expected Results}

The final result of this project is expected to be an algorithm which accepts a graph G as its input, and returns a list of possible genii as its output. It is expected that the output of this algorithm will be a list small enough such that the options could theoretically be passed, along with G itself, to an implementation of Mohar's 1999 fixed-k GGP algorithm. The algorithm will send G through a pipeline of known approaches for bounding $\gamma$(G), as described in sections IV-B and IV-C of this proposal. These approaches will provide a set of heuristics for reducing the options for $\gamma$(G) down to a tractable range. 

The results of the proposed algorithm will be compared against other approaches for determining $\gamma$(G), such as Bouchet et al. (1978) \cite{bouchet} and Kurauskas (2017) \cite{kurauskas}. While these works determing $\gamma$(G) for specific types of graphs, the algorithm from this project is expected to determine a range of possible genii for \textit{any} graph. Therefore, many known results for $\gamma$(G) will be gathered for various types of graphs for comparison between the project's algorithm and the current state-of-the-art.

The proposed algorithm will also be measured in terms of its performance and complexity. The computational complexity of the algorithm to arise from this project is not yet known. However, the algorithm itself is expected to be in polynomial time since all steps of the algorithm (maximal genus bounding, graph minor identification) are themselves in P. A key challenge to be addressed in this project is the matter of evaluating and storing a large database of graph minors for comparison against the input graph G. If the search space of graph minors is too large, then the fact that the graph minor identificaiton problem can be solved in polynomial time may become meaningless. The algorithm developed in this project must be able to limit the number of minors being searched for while scaling to a large enough database of minors in order to be practical. 

\section{Programming Components}

A key aspect of this project is the implementation of the algorithm described in the previous section. The algorithm will supply a reduced genii range for a graph G and will apply to orientable and unorientable genii. The algorithm will accept a graph G as its input, bound $\gamma$m(G) mathematically, and further bound G based on Graph Minor Theory by comparing G against a database of known and stored forbidden graph minors for various surfaces. A list of integers will be returned, corresponding to the bounded possibilities for $\gamma$(G). The following pseudocode outlines the proposed implementation of this project's algorithm:

\begin{algorithm}[H]
  \caption{Graph Genus Bounding Algorithm}
  \label{graph_genus_bound}
  \begin{algorithmic}[1]
  \State{$G \gets \{Input\ Graph\}$}
  \State{$\gamma m(G) \gets MaximalGenus(G)$}
  \State{$\gamma_{opt} \gets [\ ]\ /* List\ of\ possible\ genii */$}
  \For{($k=0\  upto\ \gamma m(G)$)}
  	\For{$H \in ForbiddenMinors(k)$}
  	\If{$G\ has\ minor\ H$}
  		\State{$exit\ search:\ move\ to\ next\ k$}
  	\EndIf
  	\EndFor
  	\State{$\gamma_{opt} \gets k$}
  	\EndFor
  	\State{$return\ \gamma_{opt}$}
  \end{algorithmic}
\end{algorithm}

Note that $\gamma$m(G) denotes the maximal genus of G based on the Betti Number, and $\gamma_{opt}$ denotes the list of possible genii options to be recorded by the algorithm. The purpose of this algorithm is to generate a list of possible genii that can be passed along with G to Mohar's 1999 fixed-k linear time GGP algorithm. Implementing Mohar's algorithm is outside the scope of the proposed project. The most significant performance bottleneck of this algorithm is expected to arise from line (5), where G is compared against every known forbidden minor H for some fixed genus k. This bottleneck will be reconciled using methods described in section IV-B to reduce the total number of forbidden minors that G must be compared against. Other heuristics may also be introduced to search for specific graph minors depending on the characteristics of the input graph G. For example: A graph G with no minor $TK_{3,3}$ is toroidal iff G does not contain any member of a specific, small set of graphs \cite{gagarin-myrvold}. Rules such as this will be capitalized upon to improve the efficiency of the forbidden minor search.

\section{Conclusion}

The aim of this project is to develop an algorithm to bound $\gamma$(G) down to a range small enough to permit the repeated execution of Mohar's 1999 fixed-k GGP algorithm. This project will use results from multiple graph theory subdisciplines to implement a pipeline of heuristics that will reduce the possible genii for an input graph G. Users of this algorithm will then be able to determine $\gamma$(G) from the possible given options by running Mohar's algorithm.

Mohar's 1999 algorithm stands as the sole method for determining if a graph G may be embedded on some surface $\Sigma$ in polynomial time. Other similar algorithms were proposed but were disproven by Myrvold et al in 2011. Implementation of Mohar's algorithm and any other algorithm for giving the embedding of a graph is outside the scope of this project. The programming aspect of this project is focused on (1) determining $\gamma$m(G) and (2) implementing and optimizing a search for forbidden minors within G. This algorithm will be implemented for both the orientable and unorientable genus of G.



\begin{thebibliography}{1}
\bibitem{filotti} Filotti, I. S., Gary L. Miller, and John Reif. On determining the genus of a graph in O (v O (g)) steps (Preliminary Report). In Proceedings of the eleventh annual ACM symposium on Theory of computing, pp. 27-37. ACM, 1979.
\bibitem{thomassen} Thomassen, Carsten. The graph genus problem is NP-complete. Journal of Algorithms 10, no. 4 (1989): 568-576.
\bibitem{myrvold-kocay} Myrvold, Wendy, and William Kocay. Errors in graph embedding algorithms. Journal of Computer and System Sciences 77, no. 2 (2011): 430-438.
\bibitem{mohar} Mohar, Bojan. A linear time algorithm for embedding graphs in an arbitrary surface. SIAM Journal on Discrete Mathematics 12, no. 1 (1999): 6-26.
\bibitem{xuong} Xuong, Nguyen Huy. How to determine the maximum genus of a graph. Journal of Combinatorial Theory, Series B 26, no. 2 (1979): 217-225.
\bibitem{gao} Kocay, William, and Donald L. Kreher. Graphs, algorithms, and optimization. CRC Press, 2016.
\bibitem{archdeacon} Archdeacon, D., Kotrbčík, M., Nedela, R. and Škoviera, M., 2014. Maximum genus, connectivity, and Nebeský's Theorem. ARS MATHEMATICA CONTEMPORANEA, 9(1).
\bibitem{gagarin-myrvold} Gagarin, Andrei, Wendy Myrvold, and John Chambers. Forbidden minors and subdivisions for toroidal graphs with no K3, 3's. Electronic Notes in Discrete Mathematics 22 (2005): 151-156.
\bibitem{thomas} Mattman, Thomas W. Forbidden minors: Finding the finite few. arXiv preprint arXiv:1608.04066 (2016).
\bibitem{lovasz} Lovasz, Laszlo. Graph minor theory. Bulletin of the American Mathematical Society 43, no. 1 (2006): 75-86.
\bibitem{graph-minors-8} Robertson, Neil, and Paul D. Seymour. Graph minors. VIII. A Kuratowski theorem for general surfaces. Journal of Combinatorial Theory, series B 48, no. 2 (1990): 255-288.
\bibitem{kocay} Kocay, William, Daniel Neilson, and Ryan Szypowski. Drawing graphs on the torus. Ars Combinatoria 59, no. 2 (2001): 259-277.
\bibitem{gagarin} Gagarin, Andrei Vladimirovich. Graph embedding algorithms. Ph.D Thesis, University of Manitoba (2003).
\bibitem{nebesky} L. Nebesky, A new characterization of the maximum genus of a graph, Czechoslovak Math. J. 31(1981), 604–613.
\bibitem{mohar-thomassen} Mohar, Bojan, and Carsten Thomassen. Graphs on surfaces. Vol. 2. Baltimore: Johns Hopkins University Press, 2001.
\bibitem{kawarabayashi} Kawarabayashi, Ken-ichi, Yusuke Kobayashi, and Bruce Reed. The disjoint paths problem in quadratic time. Journal of Combinatorial Theory, Series B 102, no. 2 (2012): 424-435.
\bibitem{graph-minors-13} Robertson, Neil, and Paul D. Seymour. Graph minors. XIII. The disjoint paths problem. Journal of combinatorial theory, Series B 63, no. 1 (1995): 65-110.
\bibitem{gagarin-kocay} Gagarin, Andrei, and William Kocay. Embedding graphs containing K5-subdivisions. Ars Comb. 64 (2002): 33.
\bibitem{bouchet} Bouchet, André. Orientable and nonorientable genus of the complete bipartite graph. Journal of Combinatorial Theory, Series B 24, no. 1 (1978): 24-33.
\bibitem{kurauskas} Kurauskas, Valentas. On the genus of the complete tripartite graph Kn, n, 1. Discrete Mathematics 340, no. 3 (2017): 508-515.
\end{thebibliography}

\end{document}